% Options for packages loaded elsewhere
\PassOptionsToPackage{unicode}{hyperref}
\PassOptionsToPackage{hyphens}{url}
%
\documentclass[
]{article}
\usepackage{amsmath,amssymb}
\usepackage{lmodern}
\usepackage{iftex}
\ifPDFTeX
  \usepackage[T1]{fontenc}
  \usepackage[utf8]{inputenc}
  \usepackage{textcomp} % provide euro and other symbols
\else % if luatex or xetex
  \usepackage{unicode-math}
  \defaultfontfeatures{Scale=MatchLowercase}
  \defaultfontfeatures[\rmfamily]{Ligatures=TeX,Scale=1}
\fi
% Use upquote if available, for straight quotes in verbatim environments
\IfFileExists{upquote.sty}{\usepackage{upquote}}{}
\IfFileExists{microtype.sty}{% use microtype if available
  \usepackage[]{microtype}
  \UseMicrotypeSet[protrusion]{basicmath} % disable protrusion for tt fonts
}{}
\makeatletter
\@ifundefined{KOMAClassName}{% if non-KOMA class
  \IfFileExists{parskip.sty}{%
    \usepackage{parskip}
  }{% else
    \setlength{\parindent}{0pt}
    \setlength{\parskip}{6pt plus 2pt minus 1pt}}
}{% if KOMA class
  \KOMAoptions{parskip=half}}
\makeatother
\usepackage{xcolor}
\usepackage[margin=1in]{geometry}
\usepackage{color}
\usepackage{fancyvrb}
\newcommand{\VerbBar}{|}
\newcommand{\VERB}{\Verb[commandchars=\\\{\}]}
\DefineVerbatimEnvironment{Highlighting}{Verbatim}{commandchars=\\\{\}}
% Add ',fontsize=\small' for more characters per line
\usepackage{framed}
\definecolor{shadecolor}{RGB}{248,248,248}
\newenvironment{Shaded}{\begin{snugshade}}{\end{snugshade}}
\newcommand{\AlertTok}[1]{\textcolor[rgb]{0.94,0.16,0.16}{#1}}
\newcommand{\AnnotationTok}[1]{\textcolor[rgb]{0.56,0.35,0.01}{\textbf{\textit{#1}}}}
\newcommand{\AttributeTok}[1]{\textcolor[rgb]{0.77,0.63,0.00}{#1}}
\newcommand{\BaseNTok}[1]{\textcolor[rgb]{0.00,0.00,0.81}{#1}}
\newcommand{\BuiltInTok}[1]{#1}
\newcommand{\CharTok}[1]{\textcolor[rgb]{0.31,0.60,0.02}{#1}}
\newcommand{\CommentTok}[1]{\textcolor[rgb]{0.56,0.35,0.01}{\textit{#1}}}
\newcommand{\CommentVarTok}[1]{\textcolor[rgb]{0.56,0.35,0.01}{\textbf{\textit{#1}}}}
\newcommand{\ConstantTok}[1]{\textcolor[rgb]{0.00,0.00,0.00}{#1}}
\newcommand{\ControlFlowTok}[1]{\textcolor[rgb]{0.13,0.29,0.53}{\textbf{#1}}}
\newcommand{\DataTypeTok}[1]{\textcolor[rgb]{0.13,0.29,0.53}{#1}}
\newcommand{\DecValTok}[1]{\textcolor[rgb]{0.00,0.00,0.81}{#1}}
\newcommand{\DocumentationTok}[1]{\textcolor[rgb]{0.56,0.35,0.01}{\textbf{\textit{#1}}}}
\newcommand{\ErrorTok}[1]{\textcolor[rgb]{0.64,0.00,0.00}{\textbf{#1}}}
\newcommand{\ExtensionTok}[1]{#1}
\newcommand{\FloatTok}[1]{\textcolor[rgb]{0.00,0.00,0.81}{#1}}
\newcommand{\FunctionTok}[1]{\textcolor[rgb]{0.00,0.00,0.00}{#1}}
\newcommand{\ImportTok}[1]{#1}
\newcommand{\InformationTok}[1]{\textcolor[rgb]{0.56,0.35,0.01}{\textbf{\textit{#1}}}}
\newcommand{\KeywordTok}[1]{\textcolor[rgb]{0.13,0.29,0.53}{\textbf{#1}}}
\newcommand{\NormalTok}[1]{#1}
\newcommand{\OperatorTok}[1]{\textcolor[rgb]{0.81,0.36,0.00}{\textbf{#1}}}
\newcommand{\OtherTok}[1]{\textcolor[rgb]{0.56,0.35,0.01}{#1}}
\newcommand{\PreprocessorTok}[1]{\textcolor[rgb]{0.56,0.35,0.01}{\textit{#1}}}
\newcommand{\RegionMarkerTok}[1]{#1}
\newcommand{\SpecialCharTok}[1]{\textcolor[rgb]{0.00,0.00,0.00}{#1}}
\newcommand{\SpecialStringTok}[1]{\textcolor[rgb]{0.31,0.60,0.02}{#1}}
\newcommand{\StringTok}[1]{\textcolor[rgb]{0.31,0.60,0.02}{#1}}
\newcommand{\VariableTok}[1]{\textcolor[rgb]{0.00,0.00,0.00}{#1}}
\newcommand{\VerbatimStringTok}[1]{\textcolor[rgb]{0.31,0.60,0.02}{#1}}
\newcommand{\WarningTok}[1]{\textcolor[rgb]{0.56,0.35,0.01}{\textbf{\textit{#1}}}}
\usepackage{graphicx}
\makeatletter
\def\maxwidth{\ifdim\Gin@nat@width>\linewidth\linewidth\else\Gin@nat@width\fi}
\def\maxheight{\ifdim\Gin@nat@height>\textheight\textheight\else\Gin@nat@height\fi}
\makeatother
% Scale images if necessary, so that they will not overflow the page
% margins by default, and it is still possible to overwrite the defaults
% using explicit options in \includegraphics[width, height, ...]{}
\setkeys{Gin}{width=\maxwidth,height=\maxheight,keepaspectratio}
% Set default figure placement to htbp
\makeatletter
\def\fps@figure{htbp}
\makeatother
\setlength{\emergencystretch}{3em} % prevent overfull lines
\providecommand{\tightlist}{%
  \setlength{\itemsep}{0pt}\setlength{\parskip}{0pt}}
\setcounter{secnumdepth}{-\maxdimen} % remove section numbering
\ifLuaTeX
  \usepackage{selnolig}  % disable illegal ligatures
\fi
\IfFileExists{bookmark.sty}{\usepackage{bookmark}}{\usepackage{hyperref}}
\IfFileExists{xurl.sty}{\usepackage{xurl}}{} % add URL line breaks if available
\urlstyle{same} % disable monospaced font for URLs
\hypersetup{
  pdftitle={Lab 2.1 - The MPG Dataset*},
  pdfauthor={Sherise Immanuela},
  hidelinks,
  pdfcreator={LaTeX via pandoc}}

\title{Lab 2.1 - The MPG Dataset*}
\author{Sherise Immanuela}
\date{Fall 2022}

\begin{document}
\maketitle

In this document, we explore the MPG dataset provided in the TidyVerse
package. It will follow the model of the lab 1.2 exploration of the
Seattle Pets dataset.

\hypertarget{loading-the-mpg-dataset}{%
\subsection{Loading the MPG Dataset}\label{loading-the-mpg-dataset}}

\begin{Shaded}
\begin{Highlighting}[]
\FunctionTok{library}\NormalTok{(tidyverse)}
\end{Highlighting}
\end{Shaded}

\begin{verbatim}
## -- Attaching packages --------------------------------------- tidyverse 1.3.1 --
\end{verbatim}

\begin{verbatim}
## v ggplot2 3.3.5     v purrr   0.3.4
## v tibble  3.1.6     v dplyr   1.0.9
## v tidyr   1.2.0     v stringr 1.4.0
## v readr   2.1.2     v forcats 0.5.1
\end{verbatim}

\begin{verbatim}
## -- Conflicts ------------------------------------------ tidyverse_conflicts() --
## x dplyr::filter() masks stats::filter()
## x dplyr::lag()    masks stats::lag()
\end{verbatim}

\begin{Shaded}
\begin{Highlighting}[]
\FunctionTok{glimpse}\NormalTok{(mpg)}
\end{Highlighting}
\end{Shaded}

\begin{verbatim}
## Rows: 234
## Columns: 11
## $ manufacturer <chr> "audi", "audi", "audi", "audi", "audi", "audi", "audi", "~
## $ model        <chr> "a4", "a4", "a4", "a4", "a4", "a4", "a4", "a4 quattro", "~
## $ displ        <dbl> 1.8, 1.8, 2.0, 2.0, 2.8, 2.8, 3.1, 1.8, 1.8, 2.0, 2.0, 2.~
## $ year         <int> 1999, 1999, 2008, 2008, 1999, 1999, 2008, 1999, 1999, 200~
## $ cyl          <int> 4, 4, 4, 4, 6, 6, 6, 4, 4, 4, 4, 6, 6, 6, 6, 6, 6, 8, 8, ~
## $ trans        <chr> "auto(l5)", "manual(m5)", "manual(m6)", "auto(av)", "auto~
## $ drv          <chr> "f", "f", "f", "f", "f", "f", "f", "4", "4", "4", "4", "4~
## $ cty          <int> 18, 21, 20, 21, 16, 18, 18, 18, 16, 20, 19, 15, 17, 17, 1~
## $ hwy          <int> 29, 29, 31, 30, 26, 26, 27, 26, 25, 28, 27, 25, 25, 25, 2~
## $ fl           <chr> "p", "p", "p", "p", "p", "p", "p", "p", "p", "p", "p", "p~
## $ class        <chr> "compact", "compact", "compact", "compact", "compact", "c~
\end{verbatim}

\begin{Shaded}
\begin{Highlighting}[]
\NormalTok{mpg}
\end{Highlighting}
\end{Shaded}

\begin{verbatim}
## # A tibble: 234 x 11
##    manufacturer model      displ  year   cyl trans drv     cty   hwy fl    class
##    <chr>        <chr>      <dbl> <int> <int> <chr> <chr> <int> <int> <chr> <chr>
##  1 audi         a4           1.8  1999     4 auto~ f        18    29 p     comp~
##  2 audi         a4           1.8  1999     4 manu~ f        21    29 p     comp~
##  3 audi         a4           2    2008     4 manu~ f        20    31 p     comp~
##  4 audi         a4           2    2008     4 auto~ f        21    30 p     comp~
##  5 audi         a4           2.8  1999     6 auto~ f        16    26 p     comp~
##  6 audi         a4           2.8  1999     6 manu~ f        18    26 p     comp~
##  7 audi         a4           3.1  2008     6 auto~ f        18    27 p     comp~
##  8 audi         a4 quattro   1.8  1999     4 manu~ 4        18    26 p     comp~
##  9 audi         a4 quattro   1.8  1999     4 auto~ 4        16    25 p     comp~
## 10 audi         a4 quattro   2    2008     4 manu~ 4        20    28 p     comp~
## # ... with 224 more rows
\end{verbatim}

\begin{Shaded}
\begin{Highlighting}[]
\FunctionTok{head}\NormalTok{(mpg, }\AttributeTok{n =} \DecValTok{1}\NormalTok{)}
\end{Highlighting}
\end{Shaded}

\begin{verbatim}
## # A tibble: 1 x 11
##   manufacturer model displ  year   cyl trans    drv     cty   hwy fl    class  
##   <chr>        <chr> <dbl> <int> <int> <chr>    <chr> <int> <int> <chr> <chr>  
## 1 audi         a4      1.8  1999     4 auto(l5) f        18    29 p     compact
\end{verbatim}

\begin{itemize}
\tightlist
\item
  This dataset has 234 rows and 11 columns.
\item
  (\href{https://ggplot2.tidyverse.org/reference/mpg.html}{MPG dataset
  reference}).
\end{itemize}

\hypertarget{reading-and-writing-files}{%
\subsection{Reading and Writing Files}\label{reading-and-writing-files}}

The MPG dataset is pre-loaded in the TidyVerse package, but we could
save it to a file and reload it from there.

We can now read the MPG dataset back in from the file.

\begin{Shaded}
\begin{Highlighting}[]
\FunctionTok{write\_csv}\NormalTok{(mpg,}\StringTok{\textquotesingle{}data/mpgdataset.csv\textquotesingle{}}\NormalTok{)}
\end{Highlighting}
\end{Shaded}

\begin{Shaded}
\begin{Highlighting}[]
\NormalTok{mpgdata }\OtherTok{\textless{}{-}} \FunctionTok{read\_csv}\NormalTok{(}\StringTok{\textquotesingle{}data/mpgdataset.csv\textquotesingle{}}\NormalTok{)}
\end{Highlighting}
\end{Shaded}

\begin{verbatim}
## Rows: 234 Columns: 11
## -- Column specification --------------------------------------------------------
## Delimiter: ","
## chr (6): manufacturer, model, trans, drv, fl, class
## dbl (5): displ, year, cyl, cty, hwy
## 
## i Use `spec()` to retrieve the full column specification for this data.
## i Specify the column types or set `show_col_types = FALSE` to quiet this message.
\end{verbatim}

\hypertarget{including-images}{%
\subsection{Including Images}\label{including-images}}

RMarkdown allows us to include images.

\includegraphics[width=1.35417in,height=\textheight]{https://cs.calvin.edu/courses/info/601/kvlinden/kvlinden/resources/images/rmarkdown-logo.png}

\hypertarget{using-equations-and-inline-code}{%
\subsection{Using Equations and Inline
Code}\label{using-equations-and-inline-code}}

RMarkdown also allows us to include equations and inline code
computations. For example, Einstein's famous equation, \(e = mc^2\),
tells us that given the speed of light (29979245800 cm/second), the
energy stored in a mass of 1 gram is:
\ensuremath{8.9875518\times 10^{20}}. (Yep, that's a pretty big number.)

The average city mpg is 16.8589744 The average highway mpg 23.4401709

\hypertarget{using-the-intermediate-data-types}{%
\subsection{Using the Intermediate Data
Types}\label{using-the-intermediate-data-types}}

This document has already includes Tibbles (e.g., \texttt{mpg}), a
particularly useful version of the standard data frame used to store
datasets. The other data types we'll use in the course include Dates and
Factors.

A \emph{date} is a special type used to represent date-times. For
example, it is currently 2022-09-12 10:20:38. Dates can be manipulated
using the \texttt{lubridate} package.

\begin{Shaded}
\begin{Highlighting}[]
\NormalTok{lubridate}\SpecialCharTok{::}\FunctionTok{make\_date}\NormalTok{(}\AttributeTok{year =}\NormalTok{ mpg}\SpecialCharTok{$}\NormalTok{year)}
\end{Highlighting}
\end{Shaded}

\begin{verbatim}
##   [1] "1999-01-01" "1999-01-01" "2008-01-01" "2008-01-01" "1999-01-01"
##   [6] "1999-01-01" "2008-01-01" "1999-01-01" "1999-01-01" "2008-01-01"
##  [11] "2008-01-01" "1999-01-01" "1999-01-01" "2008-01-01" "2008-01-01"
##  [16] "1999-01-01" "2008-01-01" "2008-01-01" "2008-01-01" "2008-01-01"
##  [21] "2008-01-01" "1999-01-01" "2008-01-01" "1999-01-01" "1999-01-01"
##  [26] "2008-01-01" "2008-01-01" "2008-01-01" "2008-01-01" "2008-01-01"
##  [31] "1999-01-01" "1999-01-01" "1999-01-01" "2008-01-01" "1999-01-01"
##  [36] "2008-01-01" "2008-01-01" "1999-01-01" "1999-01-01" "1999-01-01"
##  [41] "1999-01-01" "2008-01-01" "2008-01-01" "2008-01-01" "1999-01-01"
##  [46] "1999-01-01" "2008-01-01" "2008-01-01" "2008-01-01" "2008-01-01"
##  [51] "1999-01-01" "1999-01-01" "2008-01-01" "2008-01-01" "2008-01-01"
##  [56] "1999-01-01" "1999-01-01" "1999-01-01" "2008-01-01" "2008-01-01"
##  [61] "2008-01-01" "1999-01-01" "2008-01-01" "1999-01-01" "2008-01-01"
##  [66] "2008-01-01" "2008-01-01" "2008-01-01" "2008-01-01" "2008-01-01"
##  [71] "1999-01-01" "1999-01-01" "2008-01-01" "1999-01-01" "1999-01-01"
##  [76] "1999-01-01" "2008-01-01" "1999-01-01" "1999-01-01" "1999-01-01"
##  [81] "2008-01-01" "2008-01-01" "1999-01-01" "1999-01-01" "1999-01-01"
##  [86] "1999-01-01" "1999-01-01" "2008-01-01" "1999-01-01" "2008-01-01"
##  [91] "1999-01-01" "1999-01-01" "2008-01-01" "2008-01-01" "1999-01-01"
##  [96] "1999-01-01" "2008-01-01" "2008-01-01" "2008-01-01" "1999-01-01"
## [101] "1999-01-01" "1999-01-01" "1999-01-01" "1999-01-01" "2008-01-01"
## [106] "2008-01-01" "2008-01-01" "2008-01-01" "1999-01-01" "1999-01-01"
## [111] "2008-01-01" "2008-01-01" "1999-01-01" "1999-01-01" "2008-01-01"
## [116] "1999-01-01" "1999-01-01" "2008-01-01" "2008-01-01" "2008-01-01"
## [121] "2008-01-01" "2008-01-01" "2008-01-01" "2008-01-01" "1999-01-01"
## [126] "1999-01-01" "2008-01-01" "2008-01-01" "2008-01-01" "2008-01-01"
## [131] "1999-01-01" "2008-01-01" "2008-01-01" "1999-01-01" "1999-01-01"
## [136] "1999-01-01" "2008-01-01" "1999-01-01" "2008-01-01" "2008-01-01"
## [141] "1999-01-01" "1999-01-01" "1999-01-01" "2008-01-01" "2008-01-01"
## [146] "2008-01-01" "2008-01-01" "1999-01-01" "1999-01-01" "2008-01-01"
## [151] "1999-01-01" "1999-01-01" "2008-01-01" "2008-01-01" "1999-01-01"
## [156] "1999-01-01" "1999-01-01" "2008-01-01" "2008-01-01" "1999-01-01"
## [161] "1999-01-01" "2008-01-01" "2008-01-01" "2008-01-01" "2008-01-01"
## [166] "1999-01-01" "1999-01-01" "1999-01-01" "1999-01-01" "2008-01-01"
## [171] "2008-01-01" "2008-01-01" "2008-01-01" "1999-01-01" "1999-01-01"
## [176] "1999-01-01" "1999-01-01" "2008-01-01" "2008-01-01" "1999-01-01"
## [181] "1999-01-01" "2008-01-01" "2008-01-01" "1999-01-01" "1999-01-01"
## [186] "2008-01-01" "1999-01-01" "1999-01-01" "2008-01-01" "2008-01-01"
## [191] "1999-01-01" "1999-01-01" "2008-01-01" "1999-01-01" "1999-01-01"
## [196] "1999-01-01" "2008-01-01" "2008-01-01" "1999-01-01" "2008-01-01"
## [201] "1999-01-01" "1999-01-01" "2008-01-01" "1999-01-01" "1999-01-01"
## [206] "2008-01-01" "2008-01-01" "1999-01-01" "1999-01-01" "2008-01-01"
## [211] "2008-01-01" "1999-01-01" "1999-01-01" "1999-01-01" "1999-01-01"
## [216] "2008-01-01" "2008-01-01" "2008-01-01" "2008-01-01" "1999-01-01"
## [221] "1999-01-01" "1999-01-01" "1999-01-01" "1999-01-01" "1999-01-01"
## [226] "2008-01-01" "2008-01-01" "1999-01-01" "1999-01-01" "2008-01-01"
## [231] "2008-01-01" "1999-01-01" "1999-01-01" "2008-01-01"
\end{verbatim}

A \emph{factor} is a special type of \emph{vector} used to represent
categorical data values. For example, though the drive variable in the
MPG dataset is represented as a character, it's probably best seen as a
value from a short list of possible categories: `f', `4', \ldots{}

\begin{Shaded}
\begin{Highlighting}[]
\FunctionTok{as.factor}\NormalTok{(mpg}\SpecialCharTok{$}\NormalTok{drv)}
\end{Highlighting}
\end{Shaded}

\begin{verbatim}
##   [1] f f f f f f f 4 4 4 4 4 4 4 4 4 4 4 r r r r r r r r r r 4 4 4 4 f f f f f
##  [38] f f f f f f f f f f f 4 4 4 4 4 4 4 4 4 4 4 4 4 4 4 4 4 4 4 4 4 4 4 4 4 4
##  [75] r r r 4 4 4 4 4 4 4 4 4 4 4 4 4 r r r r r r r r r f f f f f f f f f f f f
## [112] f f f f f f f f f f f 4 4 4 4 4 4 4 4 4 4 4 4 r r r 4 4 4 4 f f f f f f f
## [149] f f 4 4 4 4 f f f f f 4 4 4 4 4 4 4 4 4 4 4 4 4 4 4 4 4 4 4 4 f f f f f f
## [186] f f f f f f f f f f f f f 4 4 4 4 4 4 4 4 4 f f f f f f f f f f f f f f f
## [223] f f f f f f f f f f f f
## Levels: 4 f r
\end{verbatim}

The variables stand for wheel drive types.

\hypertarget{exploring-the-data}{%
\subsection{Exploring the Data}\label{exploring-the-data}}

\begin{Shaded}
\begin{Highlighting}[]
\NormalTok{mpg}\SpecialCharTok{\%\textgreater{}\%}
  \FunctionTok{count}\NormalTok{(class, }\AttributeTok{sort =} \ConstantTok{TRUE}\NormalTok{)}
\end{Highlighting}
\end{Shaded}

\begin{verbatim}
## # A tibble: 7 x 2
##   class          n
##   <chr>      <int>
## 1 suv           62
## 2 compact       47
## 3 midsize       41
## 4 subcompact    35
## 5 pickup        33
## 6 minivan       11
## 7 2seater        5
\end{verbatim}

The most popular car class in the dataset is an SUV, while the least
popular is a 2 seater.

*Exercise based on \href{https://datasciencebox.org/}{Data Science in a
Box}

\end{document}
